\documentclass[journal,12pt]{IEEEtran}
\usepackage{longtable}
\usepackage{setspace}
\usepackage{gensymb}
\singlespacing
\usepackage[cmex10]{amsmath}
\newcommand\myemptypage{
	\null
	\thispagestyle{empty}
	\addtocounter{page}{-1}
	\newpage
}
\usepackage{amsthm}
\usepackage{mdframed}
\usepackage{mathrsfs}
\usepackage{txfonts}
\usepackage{stfloats}
\usepackage{bm}
\usepackage{cite}
\usepackage{cases}
\usepackage{subfig}

\usepackage{longtable}
\usepackage{multirow}

\usepackage{enumitem}
\usepackage{mathtools}
\usepackage{steinmetz}
\usepackage{tikz}
\usepackage{circuitikz}
\usepackage{verbatim}
\usepackage{tfrupee}
\usepackage[breaklinks=true]{hyperref}
\usepackage{graphicx}
\usepackage{tkz-euclide}

\usetikzlibrary{calc,math}
\usepackage{listings}
    \usepackage{color}                                            %%
    \usepackage{array}                                            %%
    \usepackage{longtable}                                        %%
    \usepackage{calc}                                             %%
    \usepackage{multirow}                                         %%
    \usepackage{hhline}                                           %%
    \usepackage{ifthen}                                           %%
    \usepackage{lscape}     
\usepackage{multicol}
\usepackage{chngcntr}

\DeclareMathOperator*{\Res}{Res}

\renewcommand\thesection{\arabic{section}}
\renewcommand\thesubsection{\thesection.\arabic{subsection}}
\renewcommand\thesubsubsection{\thesubsection.\arabic{subsubsection}}

\renewcommand\thesectiondis{\arabic{section}}
\renewcommand\thesubsectiondis{\thesectiondis.\arabic{subsection}}
\renewcommand\thesubsubsectiondis{\thesubsectiondis.\arabic{subsubsection}}


\hyphenation{op-tical net-works semi-conduc-tor}
\def\inputGnumericTable{}                                 %%

\lstset{
%language=C,
frame=single, 
breaklines=true,
columns=fullflexible
}
\begin{document}
\onecolumn

\newtheorem{theorem}{Theorem}[section]
\newtheorem{problem}{Problem}
\newtheorem{proposition}{Proposition}[section]
\newtheorem{lemma}{Lemma}[section]
\newtheorem{corollary}[theorem]{Corollary}
\newtheorem{example}{Example}[section]
\newtheorem{definition}[problem]{Definition}

\newcommand{\BEQA}{\begin{eqnarray}}
\newcommand{\EEQA}{\end{eqnarray}}
\newcommand{\define}{\stackrel{\triangle}{=}}
\bibliographystyle{IEEEtran}
\raggedbottom
\setlength{\parindent}{0pt}
\providecommand{\mbf}{\mathbf}
\providecommand{\pr}[1]{\ensuremath{\Pr\left(#1\right)}}
\providecommand{\qfunc}[1]{\ensuremath{Q\left(#1\right)}}
\providecommand{\sbrak}[1]{\ensuremath{{}\left[#1\right]}}
\providecommand{\lsbrak}[1]{\ensuremath{{}\left[#1\right.}}
\providecommand{\rsbrak}[1]{\ensuremath{{}\left.#1\right]}}
\providecommand{\brak}[1]{\ensuremath{\left(#1\right)}}
\providecommand{\lbrak}[1]{\ensuremath{\left(#1\right.}}
\providecommand{\rbrak}[1]{\ensuremath{\left.#1\right)}}
\providecommand{\cbrak}[1]{\ensuremath{\left\{#1\right\}}}
\providecommand{\lcbrak}[1]{\ensuremath{\left\{#1\right.}}
\providecommand{\rcbrak}[1]{\ensuremath{\left.#1\right\}}}
\theoremstyle{remark}
\newtheorem{rem}{Remark}
\newcommand{\sgn}{\mathop{\mathrm{sgn}}}
\providecommand{\abs}[1]{\left\vert#1\right\vert}
\providecommand{\res}[1]{\Res\displaylimits_{#1}} 
\providecommand{\norm}[1]{\left\lVert#1\right\rVert}
%\providecommand{\norm}[1]{\lVert#1\rVert}
\providecommand{\mtx}[1]{\mathbf{#1}}
\providecommand{\mean}[1]{E\left[ #1 \right]}
\providecommand{\fourier}{\overset{\mathcal{F}}{ \rightleftharpoons}}
%\providecommand{\hilbert}{\overset{\mathcal{H}}{ \rightleftharpoons}}
\providecommand{\system}{\overset{\mathcal{H}}{ \longleftrightarrow}}
	%\newcommand{\solution}[2]{\textbf{Solution:}{#1}}
\newcommand{\solution}{\noindent \textbf{Solution: }}
\newcommand{\cosec}{\,\text{cosec}\,}
\providecommand{\dec}[2]{\ensuremath{\overset{#1}{\underset{#2}{\gtrless}}}}
\newcommand{\myvec}[1]{\ensuremath{\begin{pmatrix}#1\end{pmatrix}}}
\newcommand{\mydet}[1]{\ensuremath{\begin{vmatrix}#1\end{vmatrix}}}
\numberwithin{equation}{subsection}
\makeatletter
\@addtoreset{figure}{problem}
\makeatother
\let\StandardTheFigure\thefigure
\let\vec\mathbf
\renewcommand{\thefigure}{\theproblem}
\def\putbox#1#2#3{\makebox[0in][l]{\makebox[#1][l]{}\raisebox{\baselineskip}[0in][0in]{\raisebox{#2}[0in][0in]{#3}}}}
     \def\rightbox#1{\makebox[0in][r]{#1}}
     \def\centbox#1{\makebox[0in]{#1}}
     \def\topbox#1{\raisebox{-\baselineskip}[0in][0in]{#1}}
     \def\midbox#1{\raisebox{-0.5\baselineskip}[0in][0in]{#1}}
\vspace{3cm}
\title{Assignment 11}
\author{AVVARU BHARAT - EE20MTECH11008}
\maketitle
\bigskip
\renewcommand{\thefigure}{\theenumi}
\renewcommand{\thetable}{\theenumi}
%
Download the latex-tikz codes from 
%
\begin{lstlisting}
https://github.com/Bharat437/Matrix_Theory/tree/master/Assignment11
\end{lstlisting}
\section{\textbf{Problem}}
(UGC-dec2016,73) : \\
%
Let $\vec{A}=\myvec{1 & 1\\1 & 0}$ and let $\alpha_n$ and $\beta_n$ denote the two eigenvalues of $\vec{A}^n$ such that $\abs{\alpha_n}\geq\abs{\beta_n}$.\\
Then
\begin{enumerate}
    \item $\alpha_n\rightarrow \infty$ as $n\rightarrow \infty$
    \item $\beta_n\rightarrow 0$ as $n\rightarrow \infty$
    \item $\beta_n$ is positive if n is even.
    \item $\beta_n$ is negative if n is odd.
\end{enumerate}
\section{\textbf{Solution}}
\begin{table}[ht!]
\centering
\begin{tabular}{|c|c|c|}
\hline
\textbf{Options} & \textbf{Solutions} & \textbf{True/False}\\
\hline
1.& Given&\\&$\vec{A}=\myvec{1 & 1\\1 & 0}$&\\&Now lets find the eigen values of matrix $\vec{A}$&\\&$
    \mydet{\vec{A}-\lambda\vec{I}}=0$&\\&
    $\implies \mydet{1-\lambda & 1\\1 & -\lambda}=0$&\\
    &$\implies \lambda^2-\lambda-1=0$&True\\&On solving we get 2 eigen values&\\& $\alpha_1=\frac{1+\sqrt{5}}{2}\quad\beta_1=\frac{1-\sqrt{5}}{2}$&\\&We know that if eigenvalue of $\vec{A}$ is $\lambda$ then eigenvalue of $\vec{A}^n$ is $\lambda^n$.&\\&In this problem we can say that the eigenvalues $\alpha_n$ and $\beta_n$ of $\vec{A}^n$ are&\\&$\alpha_n=\alpha_1^n\quad\beta_n=\beta_1^n$&\\&Since $\alpha_1>1$ we can say that $\alpha_n\rightarrow \infty$ as $n\rightarrow \infty$.&\\
\hline
2.& We got $\beta_1=\frac{1-\sqrt{5}}{2}$ and $\beta_n=\beta_1^n$. &\\&Since $-1<\beta_1<0$,we can say that $\beta_n\rightarrow 0$ as $n\rightarrow \infty$.&True\\
\hline
3.&We got $\beta_1=\frac{1-\sqrt{5}}{2}$ and $\beta_n=\beta_1^n$.&\\&Since $\beta_1$ is negative because $-1<\beta_1<0$, if n is even then $\beta_n$ is positive.&True\\
\hline
4.&We got $\beta_1=\frac{1-\sqrt{5}}{2}$ and $\beta_n=\beta_1^n$.&\\&Since $\beta_1$ is negative, if n is odd then $\beta_n$ is negative.&True\\
\hline
\end{tabular}
\end{table}
\end{document}
