\documentclass[journal,12pt,twocolumn]{IEEEtran}

\usepackage{setspace}
\usepackage{gensymb}
\usepackage{enumitem}
\singlespacing


\usepackage[cmex10]{amsmath}

\usepackage{amsthm}

\usepackage{mathrsfs}
\usepackage{txfonts}
\usepackage{stfloats}
\usepackage{bm}
\usepackage{cite}
\usepackage{cases}
\usepackage{subfig}

\usepackage{longtable}
\usepackage{multirow}

\usepackage{enumitem}
\usepackage{mathtools}
\usepackage{steinmetz}
\usepackage{tikz}
\usepackage{circuitikz}
\usepackage{verbatim}
\usepackage{tfrupee}
\usepackage[breaklinks=true]{hyperref}

\usepackage{tkz-euclide}

\usetikzlibrary{calc,math}
\usepackage{listings}
    \usepackage{color}                                            %%
    \usepackage{array}                                            %%
    \usepackage{longtable}                                        %%
    \usepackage{calc}                                             %%
    \usepackage{multirow}                                         %%
    \usepackage{hhline}                                           %%
    \usepackage{ifthen}                                           %%
    \usepackage{lscape}     
\usepackage{multicol}
\usepackage{chngcntr}

\DeclareMathOperator*{\Res}{Res}

\renewcommand\thesection{\arabic{section}}
\renewcommand\thesubsection{\thesection.\arabic{subsection}}
\renewcommand\thesubsubsection{\thesubsection.\arabic{subsubsection}}

\renewcommand\thesectiondis{\arabic{section}}
\renewcommand\thesubsectiondis{\thesectiondis.\arabic{subsection}}
\renewcommand\thesubsubsectiondis{\thesubsectiondis.\arabic{subsubsection}}


\hyphenation{op-tical net-works semi-conduc-tor}
\def\inputGnumericTable{}                                 %%

\lstset{
%language=C,
frame=single, 
breaklines=true,
columns=fullflexible
}
\begin{document}


\newtheorem{theorem}{Theorem}[section]
\newtheorem{problem}{Problem}
\newtheorem{proposition}{Proposition}[section]
\newtheorem{lemma}{Lemma}[section]
\newtheorem{corollary}[theorem]{Corollary}
\newtheorem{example}{Example}[section]
\newtheorem{definition}[problem]{Definition}

\newcommand{\BEQA}{\begin{eqnarray}}
\newcommand{\EEQA}{\end{eqnarray}}
\newcommand{\define}{\stackrel{\triangle}{=}}
\bibliographystyle{IEEEtran}
\providecommand{\mbf}{\mathbf}
\providecommand{\pr}[1]{\ensuremath{\Pr\left(#1\right)}}
\providecommand{\qfunc}[1]{\ensuremath{Q\left(#1\right)}}
\providecommand{\sbrak}[1]{\ensuremath{{}\left[#1\right]}}
\providecommand{\lsbrak}[1]{\ensuremath{{}\left[#1\right.}}
\providecommand{\rsbrak}[1]{\ensuremath{{}\left.#1\right]}}
\providecommand{\brak}[1]{\ensuremath{\left(#1\right)}}
\providecommand{\lbrak}[1]{\ensuremath{\left(#1\right.}}
\providecommand{\rbrak}[1]{\ensuremath{\left.#1\right)}}
\providecommand{\cbrak}[1]{\ensuremath{\left\{#1\right\}}}
\providecommand{\lcbrak}[1]{\ensuremath{\left\{#1\right.}}
\providecommand{\rcbrak}[1]{\ensuremath{\left.#1\right\}}}
\theoremstyle{remark}
\newtheorem{rem}{Remark}
\newcommand{\sgn}{\mathop{\mathrm{sgn}}}
\providecommand{\abs}[1]{\left\vert#1\right\vert}
\providecommand{\res}[1]{\Res\displaylimits_{#1}} 
\providecommand{\norm}[1]{\left\lVert#1\right\rVert}
%\providecommand{\norm}[1]{\lVert#1\rVert}
\providecommand{\mtx}[1]{\mathbf{#1}}
\providecommand{\mean}[1]{E\left[ #1 \right]}
\providecommand{\fourier}{\overset{\mathcal{F}}{ \rightleftharpoons}}
%\providecommand{\hilbert}{\overset{\mathcal{H}}{ \rightleftharpoons}}
\providecommand{\system}{\overset{\mathcal{H}}{ \longleftrightarrow}}
	%\newcommand{\solution}[2]{\textbf{Solution:}{#1}}
\newcommand{\solution}{\noindent \textbf{Solution: }}
\newcommand{\cosec}{\,\text{cosec}\,}
\providecommand{\dec}[2]{\ensuremath{\overset{#1}{\underset{#2}{\gtrless}}}}
\newcommand{\myvec}[1]{\ensuremath{\begin{pmatrix}#1\end{pmatrix}}}
\newcommand{\mydet}[1]{\ensuremath{\begin{vmatrix}#1\end{vmatrix}}}
\numberwithin{equation}{subsection}
\makeatletter
\@addtoreset{figure}{problem}
\makeatother
\let\StandardTheFigure\thefigure
\let\vec\mathbf
\renewcommand{\thefigure}{\theproblem}
\def\putbox#1#2#3{\makebox[0in][l]{\makebox[#1][l]{}\raisebox{\baselineskip}[0in][0in]{\raisebox{#2}[0in][0in]{#3}}}}
     \def\rightbox#1{\makebox[0in][r]{#1}}
     \def\centbox#1{\makebox[0in]{#1}}
     \def\topbox#1{\raisebox{-\baselineskip}[0in][0in]{#1}}
     \def\midbox#1{\raisebox{-0.5\baselineskip}[0in][0in]{#1}}
\vspace{3cm}
\title{Assignment 12}
\author{AVVARU BHARAT - EE20MTECH11008}
\maketitle
\newpage
\bigskip
\renewcommand{\thefigure}{\theenumi}
\renewcommand{\thetable}{\theenumi}
Download latex-tikz codes from 
%
\begin{lstlisting}
https://github.com/Bharat437/Matrix_Theory/tree/master/Assignment12
\end{lstlisting}
%
Download Python codes from 
%
\begin{lstlisting}
https://github.com/Bharat437/Matrix_Theory/tree/master/Assignment12/Codes
\end{lstlisting}
\section{Problem}
(Hoffman, page 208, 4) : \\
%
Let $\vec{A}$,$\vec{B}$,$\vec{C}$,$\vec{D}$ be $n\times n$ complex matrices which commute. Let $\vec{E}$ be the $2n\times 2n$ matrix
\begin{align}
    \vec{E}=\myvec{\vec{A} & \vec{B}\\\vec{C} & \vec{D}}\label{E}
\end{align}
prove that $\det(\vec{E})=\det(\vec{A}\vec{D}-\vec{B}\vec{C})$
\section{Solution}
Given matrices $\vec{A}$,$\vec{B}$,$\vec{C}$,$\vec{D}$ commute.

Let $\vec{P}$ be an invertible matrix that can simultaneously diagonalize matrices $\vec{A}$,$\vec{B}$,$\vec{C}$,$\vec{D}$ as below
\begin{align}
    \vec{A}=\vec{P}\vec{\Lambda_a}\vec{P}^{-1}\label{A}\\
    \vec{B}=\vec{P}\vec{\Lambda_b}\vec{P}^{-1}\label{B}\\
    \vec{C}=\vec{P}\vec{\Lambda_c}\vec{P}^{-1}\label{C}\\
    \vec{D}=\vec{P}\vec{\Lambda_d}\vec{P}^{-1}\label{D}
\end{align}
where $\vec{\Lambda_a}$,$\vec{\Lambda_b}$,$\vec{\Lambda_c}$,$\vec{\Lambda_d}$ are diagonal matrices whose diagonal values are eigenvalues of matrices $\vec{A}$,$\vec{B}$,$\vec{C}$,$\vec{D}$ respectively and matrix $\vec{P}$ is formed by n-linearly independent eigen vectors.

Now \eqref{E} can be written as
\begin{align}
    \vec{E}=\myvec{\vec{P}\vec{\Lambda_a}\vec{P}^{-1} & \vec{P}\vec{\Lambda_b}\vec{P}^{-1}\\\vec{P}\vec{\Lambda_c}\vec{P}^{-1} & \vec{P}\vec{\Lambda_d}\vec{P}^{-1}}
\end{align}
Using block matrix multiplication, we get
\begin{align}
    \implies&\vec{E}=\myvec{\vec{P} & \vec{0}\\\vec{0} & \vec{P}}\myvec{\vec{\Lambda_a} & \vec{\Lambda_b}\\\vec{\Lambda_c} & \vec{\Lambda_d}}\myvec{\vec{P}^{-1} & \vec{0}\\\vec{0} & \vec{P}^{-1}}\\
    \implies&\vec{E}=\vec{M}\vec{D}\vec{M}^{-1}\label{Edec}
\end{align}
where
\begin{align}
    \vec{M}=\myvec{\vec{P} & \vec{0}\\\vec{0} & \vec{P}}\\
    \vec{D}=\myvec{\vec{\Lambda_a} & \vec{\Lambda_b}\\\vec{\Lambda_c} & \vec{\Lambda_d}}
\end{align}
Now we will calculate $\det(\vec{E})$,
\begin{align}
    &\mydet{\vec{E}}=\mydet{\vec{M}\vec{D}\vec{M}^{-1}}\\
    \implies&\mydet{\vec{E}}=\mydet{\vec{M}}\mydet{\vec{D}}\mydet{\vec{M}^{-1}}\\
    \implies&\mydet{\vec{E}}=\mydet{\vec{M}}\mydet{\vec{D}}\mydet{\vec{M}}^{-1}\\
    \implies&\mydet{\vec{E}}=\mydet{\vec{D}}\label{EeqD}\\
    \implies&\mydet{\vec{E}}=\mydet{\vec{\Lambda_a} & \vec{\Lambda_b}\\\vec{\Lambda_c} & \vec{\Lambda_d}}\\
    &=\mydet{\lambda_{1a}&0&\dots&0&\lambda_{1b}&0&\dots&0\\0&\lambda_{2a}&\dots&0&0&\lambda_{2b}&\dots&0\\\vdots&\vdots&\dots&\vdots&\vdots&\vdots&\dots&\vdots\\0&0&\dots&\lambda_{na}&0&0&\dots&\lambda_{nb}\\\lambda_{1c}&0&\dots&0&\lambda_{1d}&0&\dots&0\\0&\lambda_{2c}&\dots&0&0&\lambda_{2d}&\dots&0\\\vdots&\vdots&\dots&\vdots&\vdots&\vdots&\dots&\vdots\\0&0&\dots&\lambda_{nc}&0&0&\dots&\lambda_{nd}}
\end{align}
Using row reduction,
\begin{multline}
    \xleftrightarrow[]{R_{n+1}=R_{n+1}-\frac{\lambda_{1c}}{\lambda_{1a}}R_1}\\
    \mydet{\lambda_{1a}&0&\dots&0&\lambda_{1b}&0&\dots&0\\0&\lambda_{2a}&\dots&0&0&\lambda_{2b}&\dots&0\\\vdots&\vdots&\dots&\vdots&\vdots&\vdots&\dots&\vdots\\0&0&\dots&\lambda_{na}&0&0&\dots&\lambda_{nb}\\0&0&\dots&0&\lambda_{1d}-\frac{\lambda_{1c}\lambda_{1b}}{\lambda_{1a}}&0&\dots&0\\0&\lambda_{2c}&\dots&0&0&\lambda_{2d}&\dots&0\\\vdots&\vdots&\dots&\vdots&\vdots&\vdots&\dots&\vdots\\0&0&\dots&\lambda_{nc}&0&0&\dots&\lambda_{nd}}
\end{multline}
similarly doing elementary row operations for rows $R_{n+2}$ to $R_{2n}$, we get

\begin{multline}
    \mydet{\vec{E}}=\\\mydet{\lambda_{1a}&\dots&0&\lambda_{1b}&\dots&0\\\vdots&\dots&\vdots&\vdots&\dots&\vdots\\0&\dots&\lambda_{na}&0&\dots&\lambda_{nb}\\0&\dots&0&\lambda_{1d}-\frac{\lambda_{1c}\lambda_{1b}}{\lambda_{1a}}&\dots&0\\\vdots&\dots&\vdots&\vdots&\dots&\vdots\\0&\dots&0&0&\dots&\lambda_{nd}-\frac{\lambda_{nc}\lambda_{nb}}{\lambda_{na}}}
\end{multline}
Since it is upper triangular matrix, then $\mydet{\vec{E}}$ will be multiplication of diagonal elements.
\begin{multline}
    \implies\mydet{\vec{E}}=\lambda_{1a}\lambda_{2a}\dots\lambda_{na}\\\times\left(\lambda_{1d}-\frac{\lambda_{1c}\lambda_{1b}}{\lambda_{1a}}\right)\dots\left(\lambda_{nd}-\frac{\lambda_{nc}\lambda_{nb}}{\lambda_{na}}\right)
\end{multline}
\begin{multline}
    \implies\mydet{\vec{E}}=(\lambda_{1a}\lambda_{1d}-\lambda_{1c}\lambda_{1b})\\\times(\lambda_{2a}\lambda_{2d}-\lambda_{2c}\lambda_{2b})\dots(\lambda_{na}\lambda_{nd}-\lambda_{nc}\lambda_{nb})\label{detE}
\end{multline}

Now we will calculate $\det(\vec{A}\vec{D}-\vec{B}\vec{C})$ by substituting \eqref{A},\eqref{B},\eqref{C},\eqref{D}
\begin{align}
    \mydet{\vec{A}\vec{D}-\vec{B}\vec{C}}&=\mydet{\vec{P}\vec{\Lambda_a}\vec{P}^{-1}\vec{P}\vec{\Lambda_d}\vec{P}^{-1}-\vec{P}\vec{\Lambda_b}\vec{P}^{-1}\vec{P}\vec{\Lambda_c}\vec{P}^{-1}}\\
    &=\mydet{\vec{P}\vec{\Lambda_a}\vec{\Lambda_d}\vec{P}^{-1}-\vec{P}\vec{\Lambda_b}\vec{\Lambda_c}\vec{P}^{-1}}\\
    &=\mydet{\vec{P}(\vec{\Lambda_a}\vec{\Lambda_d}-\vec{\Lambda_b}\vec{\Lambda_c})\vec{P}^{-1}}\\
    &=\mydet{\vec{P}}\mydet{\vec{\Lambda_a}\vec{\Lambda_d}-\vec{\Lambda_b}\vec{\Lambda_c}}\mydet{\vec{P}^{-1}}\\
    &=\mydet{\vec{P}}\mydet{\vec{P}}^{-1}\mydet{\vec{\Lambda_a}\vec{\Lambda_d}-\vec{\Lambda_b}\vec{\Lambda_c}}\\
    \mydet{\vec{A}\vec{D}-\vec{B}\vec{C}}&=\mydet{\vec{\Lambda_a}\vec{\Lambda_d}-\vec{\Lambda_b}\vec{\Lambda_c}}\label{adbc1}
\end{align}

Since $\vec{\Lambda_a}$,$\vec{\Lambda_b}$,$\vec{\Lambda_c}$,$\vec{\Lambda_d}$ are diagonal matrices, we get
\begin{align}
    \vec{\Lambda_a}\vec{\Lambda_d}=\myvec{\lambda_{1a}\lambda_{1d}&0&\dots&0\\0&\lambda_{2a}\lambda_{2d}&\dots&0\\\vdots&\vdots&\dots&\vdots\\0&0&\dots&\lambda_{na}\lambda_{nd}}\label{lald}\\
    \vec{\Lambda_b}\vec{\Lambda_c}=\myvec{\lambda_{1b}\lambda_{1c}&0&\dots&0\\0&\lambda_{2b}\lambda_{2c}&\dots&0\\\vdots&\vdots&\dots&\vdots\\0&0&\dots&\lambda_{nb}\lambda_{nc}}\label{lblc}\\
\end{align}

Substitute \eqref{lald} and \eqref{lblc} in \eqref{adbc1}, we get
\begin{multline}
    \mydet{\vec{A}\vec{D}-\vec{B}\vec{C}}=\\\mydet{\lambda_{1a}\lambda_{1d}-\lambda_{1b}\lambda_{1c}&\dots&0\\0&\dots&0\\\vdots&\dots&\vdots\\0&\dots&\lambda_{na}\lambda_{nd}-\lambda_{nb}\lambda_{nc}}
\end{multline}
\begin{multline}
    \implies\mydet{\vec{A}\vec{D}-\vec{B}\vec{C}}=(\lambda_{1a}\lambda_{1d}-\lambda_{1b}\lambda_{1c})\\\times(\lambda_{2a}\lambda_{2d}-\lambda_{2b}\lambda_{2c})\dots(\lambda_{na}\lambda_{nd}-\lambda_{nb}\lambda_{nc})\label{adbcfin}
\end{multline}

Comparing \eqref{detE} and \eqref{adbcfin} we can say that
\begin{align}
    \mydet{\vec{E}}=\mydet{\vec{A}\vec{D}-\vec{B}\vec{C}}
\end{align}
Hence proved.
\section{\textbf{Example}}
Let us consider below matrices $\vec{A}$,$\vec{B}$,$\vec{C}$,$\vec{D}$ be $n\times n$ complex matrices which commute.
\begin{align}
    \vec{A}=\myvec{1-i&-2\\3&-1-i}\label{exa}\\
    \vec{B}=\myvec{3-3i&-6\\9&-3-3i}\label{exb}\\
    \vec{C}=\myvec{7-7i&-14\\21&-7-7i}\label{exc}\\
    \vec{D}=\myvec{-2+2i&4\\-6&2+2i}\label{exd}
\end{align}
Lets find eigenvalues of matrix $\vec{A}$
\begin{align}
    &\mydet{\vec{A}-\lambda\vec{I}}=0\\
    \implies&\mydet{1-i-\lambda&-2\\3&-1-i-\lambda}=0\\
    \implies&\lambda^2+2\lambda i+4=0\\
    \implies&\lambda_{1a}=-(1+\sqrt{5})i\quad\lambda_{2a}=(-1+\sqrt{5})i\label{eiga}
\end{align}
\eqref{eiga} are eigenvalues of matrix $\vec{A}$.

The eigenvectors of matrix $\vec{A}$ are
\begin{align}
    \vec{v_1}=\myvec{\frac{-1+\sqrt{5}i}{3}\\1}\quad\vec{v_2}=\myvec{\frac{-1-\sqrt{5}i}{3}\\1}
\end{align}
Since matrices $\vec{A}$,$\vec{B}$,$\vec{C}$,$\vec{D}$ commute, from \eqref{exa}, \eqref{exb}, \eqref{exc}, \eqref{exd} we can say that
\begin{align}
    \vec{B}&=3\vec{A}\\
    \vec{C}&=7\vec{A}\\
    \vec{D}&=-2\vec{A}
\end{align}

Then the eigenvalues of matrices $\vec{B}$,$\vec{C}$,$\vec{D}$ are
\begin{align}
    \lambda_{1b}=-3(1+\sqrt{5})i\quad\lambda_{2b}=3(-1+\sqrt{5})i\label{eigb}\\
    \lambda_{1c}=-7(1+\sqrt{5})i\quad\lambda_{2c}=7(-1+\sqrt{5})i\label{eigc}\\
    \lambda_{1d}=2(1+\sqrt{5}+1)i\quad\lambda_{2d}=-2(-1+\sqrt{5})i\label{eigd}
\end{align}
But the eigenvectors of matrices $\vec{B}$,$\vec{C}$,$\vec{D}$ are same as of matrix $\vec{A}$.\\
The eigenvalue decomposition of matrices $\vec{A}$,$\vec{B}$,$\vec{C}$,$\vec{D}$ is done as in \eqref{A}, \eqref{B}, \eqref{C}, \eqref{D}. Here eigenvector matrix $\vec{P}$ and $\vec{\Lambda_a}$,$\vec{\Lambda_b}$,$\vec{\Lambda_c}$,$\vec{\Lambda_d}$ are
\begin{align}
    \vec{P}=\myvec{\frac{-1+\sqrt{5}i}{3}&\frac{-1-\sqrt{5}i}{3}\\1&1}\\
    \vec{\Lambda_a}=\myvec{mi&0\\0&ni}\label{exla}\\
    \vec{\Lambda_b}=\myvec{3mi&0\\0&3ni}\label{exlb}\\
    \vec{\Lambda_c}=\myvec{7mi&0\\0&7ni}\label{exlc}\\
    \vec{\Lambda_d}=\myvec{-2mi&0\\0&-2ni}\label{exld}
\end{align}
where
\begin{align}
    m=-(1+\sqrt{5})\quad n=(-1+\sqrt{5})\label{mn}
\end{align}
From \eqref{Edec}, we got
\begin{align}
    \vec{E}=\vec{M}\vec{D}\vec{M}^{-1}
\end{align}
where
\begin{align}
    \vec{M}&=\myvec{\vec{P} & \vec{0}\\\vec{0} & \vec{P}}\\
    &=\myvec{\frac{-1+\sqrt{5}i}{3}&\frac{-1-\sqrt{5}i}{3}&0&0\\1&1&0&0\\0&0&\frac{-1+\sqrt{5}i}{3}&\frac{-1-\sqrt{5}i}{3}\\0&0&1&1}
\end{align}
\begin{multline}
    \vec{D}=\myvec{\vec{\Lambda_a} & \vec{\Lambda_b}\\\vec{\Lambda_c} & \vec{\Lambda_d}}\\
    =\myvec{mi&0&3mi&0\\0&ni&0&3ni\\7mi&0&-2mi&0\\0&7ni&0&-2ni}
\end{multline}
Now we will calculate $\det(E)$, from \eqref{EeqD} we got
\begin{align}
    \mydet{\vec{E}}&=\mydet{\vec{D}}\\
    &=\mydet{mi&0&3mi&0\\0&ni&0&3ni\\7mi&0&-2mi&0\\0&7ni&0&-2ni}
\end{align}
Using row reduction technique,
\begin{align}
    \mydet{mi&0&3mi&0\\0&ni&0&3ni\\7mi&0&-2mi&0\\0&7ni&0&-2ni}\\
    \xleftrightarrow[]{R_3\leftarrow R_3-7R_1}\mydet{mi&0&3mi&0\\0&ni&0&3ni\\0&0&-23mi&0\\0&7ni&0&-2ni}\\
    \xleftrightarrow[]{R_4\leftarrow R_4-7R_2}\mydet{mi&0&3mi&0\\0&ni&0&3ni\\0&0&-23mi&0\\0&0&0&-23ni}
\end{align}
Now
\begin{align}
    \mydet{\vec{E}}=\mydet{mi&0&3mi&0\\0&ni&0&3ni\\0&0&-23mi&0\\0&0&0&-23ni}\\
    \implies\mydet{\vec{E}}=529m^2n^2\label{exdet1}
\end{align}
Substitute \eqref{mn} in \eqref{exdet1}, we get
\begin{align}
    \mydet{\vec{E}}=529(1+\sqrt{5})^2(-1+\sqrt{5})^2=8464\label{exdetfin}
\end{align}
Now we will calculate $\det(\vec{A}\vec{D}-\vec{B}\vec{C})$, from \eqref{adbc1} we got
\begin{align}
    \mydet{\vec{A}\vec{D}-\vec{B}\vec{C}}&=\mydet{\vec{\Lambda_a}\vec{\Lambda_d}-\vec{\Lambda_b}\vec{\Lambda_c}}\label{exadbc1}
\end{align}
Using \eqref{exla}, \eqref{exlb}, \eqref{exlc}, \eqref{exld} we get
\begin{align}
    \vec{\Lambda_a}\vec{\Lambda_d}&=\myvec{mi&0\\0&ni}\myvec{-2mi&0\\0&-2ni}\\
    \implies\vec{\Lambda_a}\vec{\Lambda_d}&=\myvec{2m^2&0\\0&2n^2}\label{exlald}\\
    \vec{\Lambda_b}\vec{\Lambda_c}&=\myvec{3mi&0\\0&3ni}\myvec{7mi&0\\0&7ni}\\
    \implies\vec{\Lambda_b}\vec{\Lambda_c}&=\myvec{-21m^2&0\\0&-21n^2}\label{exlblc}
\end{align}
Substitute \eqref{exlald} and \eqref{exlblc} in \eqref{exadbc1}, we get
\begin{align}
    \mydet{\vec{A}\vec{D}-\vec{B}\vec{C}}&=\mydet{\myvec{2m^2&0\\0&2n^2}-\myvec{-21m^2&0\\0&-21n^2}}\\
    &=\mydet{23m^2&0\\0&23n^2}\\
    &=529m^2n^2\label{exadbc1}
\end{align}
Substitute \eqref{mn} in \eqref{exadbc1}, we get
\begin{multline}
    \mydet{\vec{A}\vec{D}-\vec{B}\vec{C}}=529(1+\sqrt{5})^2(-1+\sqrt{5})^2\\=8464\label{exadbcfin}
\end{multline}
Comparing \eqref{exdetfin} and \eqref{exadbcfin}, we get
\begin{align}
    \mydet{\vec{E}}=\mydet{\vec{A}\vec{D}-\vec{B}\vec{C}}
\end{align}
\end{document}
