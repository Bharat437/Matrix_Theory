\documentclass[journal,12pt]{IEEEtran}
\usepackage{longtable}
\usepackage{setspace}
\usepackage{gensymb}
\singlespacing
\usepackage[cmex10]{amsmath}
\newcommand\myemptypage{
	\null
	\thispagestyle{empty}
	\addtocounter{page}{-1}
	\newpage
}
\usepackage{amsthm}
\usepackage{mdframed}
\usepackage{mathrsfs}
\usepackage{txfonts}
\usepackage{stfloats}
\usepackage{bm}
\usepackage{cite}
\usepackage{cases}
\usepackage{subfig}

\usepackage{longtable}
\usepackage{multirow}

\usepackage{enumitem}
\usepackage{mathtools}
\usepackage{steinmetz}
\usepackage{tikz}
\usepackage{circuitikz}
\usepackage{verbatim}
\usepackage{tfrupee}
\usepackage[breaklinks=true]{hyperref}
\usepackage{graphicx}
\usepackage{tkz-euclide}

\usetikzlibrary{calc,math}
\usepackage{listings}
    \usepackage{color}                                            %%
    \usepackage{array}                                            %%
    \usepackage{longtable}                                        %%
    \usepackage{calc}                                             %%
    \usepackage{multirow}                                         %%
    \usepackage{hhline}                                           %%
    \usepackage{ifthen}                                           %%
    \usepackage{lscape}     
\usepackage{multicol}
\usepackage{chngcntr}

\DeclareMathOperator*{\Res}{Res}

\renewcommand\thesection{\arabic{section}}
\renewcommand\thesubsection{\thesection.\arabic{subsection}}
\renewcommand\thesubsubsection{\thesubsection.\arabic{subsubsection}}

\renewcommand\thesectiondis{\arabic{section}}
\renewcommand\thesubsectiondis{\thesectiondis.\arabic{subsection}}
\renewcommand\thesubsubsectiondis{\thesubsectiondis.\arabic{subsubsection}}


\hyphenation{op-tical net-works semi-conduc-tor}
\def\inputGnumericTable{}                                 %%

\lstset{
%language=C,
frame=single, 
breaklines=true,
columns=fullflexible
}
\begin{document}
\onecolumn

\newtheorem{theorem}{Theorem}[section]
\newtheorem{problem}{Problem}
\newtheorem{proposition}{Proposition}[section]
\newtheorem{lemma}{Lemma}[section]
\newtheorem{corollary}[theorem]{Corollary}
\newtheorem{example}{Example}[section]
\newtheorem{definition}[problem]{Definition}

\newcommand{\BEQA}{\begin{eqnarray}}
\newcommand{\EEQA}{\end{eqnarray}}
\newcommand{\define}{\stackrel{\triangle}{=}}
\bibliographystyle{IEEEtran}
\raggedbottom
\setlength{\parindent}{0pt}
\providecommand{\mbf}{\mathbf}
\providecommand{\pr}[1]{\ensuremath{\Pr\left(#1\right)}}
\providecommand{\qfunc}[1]{\ensuremath{Q\left(#1\right)}}
\providecommand{\sbrak}[1]{\ensuremath{{}\left[#1\right]}}
\providecommand{\lsbrak}[1]{\ensuremath{{}\left[#1\right.}}
\providecommand{\rsbrak}[1]{\ensuremath{{}\left.#1\right]}}
\providecommand{\brak}[1]{\ensuremath{\left(#1\right)}}
\providecommand{\lbrak}[1]{\ensuremath{\left(#1\right.}}
\providecommand{\rbrak}[1]{\ensuremath{\left.#1\right)}}
\providecommand{\cbrak}[1]{\ensuremath{\left\{#1\right\}}}
\providecommand{\lcbrak}[1]{\ensuremath{\left\{#1\right.}}
\providecommand{\rcbrak}[1]{\ensuremath{\left.#1\right\}}}
\theoremstyle{remark}
\newtheorem{rem}{Remark}
\newcommand{\sgn}{\mathop{\mathrm{sgn}}}
\providecommand{\abs}[1]{\left\vert#1\right\vert}
\providecommand{\res}[1]{\Res\displaylimits_{#1}} 
\providecommand{\norm}[1]{\left\lVert#1\right\rVert}
%\providecommand{\norm}[1]{\lVert#1\rVert}
\providecommand{\mtx}[1]{\mathbf{#1}}
\providecommand{\mean}[1]{E\left[ #1 \right]}
\providecommand{\fourier}{\overset{\mathcal{F}}{ \rightleftharpoons}}
%\providecommand{\hilbert}{\overset{\mathcal{H}}{ \rightleftharpoons}}
\providecommand{\system}{\overset{\mathcal{H}}{ \longleftrightarrow}}
	%\newcommand{\solution}[2]{\textbf{Solution:}{#1}}
\newcommand{\solution}{\noindent \textbf{Solution: }}
\newcommand{\cosec}{\,\text{cosec}\,}
\providecommand{\dec}[2]{\ensuremath{\overset{#1}{\underset{#2}{\gtrless}}}}
\newcommand{\myvec}[1]{\ensuremath{\begin{pmatrix}#1\end{pmatrix}}}
\newcommand{\mydet}[1]{\ensuremath{\begin{vmatrix}#1\end{vmatrix}}}
\numberwithin{equation}{subsection}
\makeatletter
\@addtoreset{figure}{problem}
\makeatother
\let\StandardTheFigure\thefigure
\let\vec\mathbf
\renewcommand{\thefigure}{\theproblem}
\def\putbox#1#2#3{\makebox[0in][l]{\makebox[#1][l]{}\raisebox{\baselineskip}[0in][0in]{\raisebox{#2}[0in][0in]{#3}}}}
     \def\rightbox#1{\makebox[0in][r]{#1}}
     \def\centbox#1{\makebox[0in]{#1}}
     \def\topbox#1{\raisebox{-\baselineskip}[0in][0in]{#1}}
     \def\midbox#1{\raisebox{-0.5\baselineskip}[0in][0in]{#1}}
\vspace{3cm}
\title{Assignment 15}
\author{AVVARU BHARAT - EE20MTECH11008}
\maketitle
\bigskip
\renewcommand{\thefigure}{\theenumi}
\renewcommand{\thetable}{\theenumi}
%
Download the latex-tikz codes from 
%
\begin{lstlisting}
https://github.com/Bharat437/Matrix_Theory/tree/master/Assignment15
\end{lstlisting}
\section{\textbf{Problem}}
(UGC,JUNE 2015,68) : \\
%
Let $\vec{F}:\mathbb{R}^n\times\mathbb{R}^n\rightarrow\mathbb{R}$ be the function $\vec{F}(\vec{x},\vec{y})=\langle\vec{Ax},\vec{y}\rangle$, where $\langle,\rangle$ is the standard inner product of $\mathbb{R}^n$ and $\vec{A}$ is a $n\times n$ real matrix. Here D denotes the total derivative. Which of the following statements are correct?
\begin{enumerate}
    \item $(D\vec{F}(\vec{x},\vec{y}))(\vec{u},\vec{v})=\langle\vec{Au},\vec{y}\rangle+\langle\vec{Ax},\vec{v}\rangle$.
    \item $(D\vec{F}(\vec{x},\vec{y}))(0,0)=0$.
    \item $D\vec{F}(\vec{x},\vec{y})$ may not exist for some $(\vec{x},\vec{y})\in\mathbb{R}^n\times\mathbb{R}^n$.
    \item $D\vec{F}(\vec{x},\vec{y})$ does not exist at $(\vec{x},\vec{y})=(0,0)$.
\end{enumerate}
\section{\textbf{Explanation}}
\renewcommand{\thetable}{1}
\begin{longtable}{|l|l|}
\hline
\endhead
\textbf{Inner product}&Inner product between two vectors $\vec{x}$ and $\vec{y}$ is defined as\\&\parbox{13cm}{\begin{align}
    \langle\vec{x},\vec{y}\rangle=\vec{x}^T\vec{y}\label{inp}
\end{align}}\\&Where $\vec{x}$,$\vec{y}\in\mathbb{R}^n$\\
\hline
\textbf{Inner Product}&\\\textbf{Property used}&\parbox{13cm}{\begin{align}
    \langle\vec{x},\vec{y}\rangle=\vec{x}^T\vec{y}=\vec{y}^T\vec{x}=\langle\vec{y},\vec{x}\rangle\label{prop1}
    \end{align}}\\
\hline
\textbf{Total Derivative} $D$&Total derivative is a linear transformation. For function $\vec{F}(\vec{x},\vec{y})$, the total\\& derivative is given as $D\vec{F}(\vec{x},\vec{y})$ which says that total derivative of\\&function $\vec{F}$ at $(\vec{x},\vec{y})$.\\
\hline
\caption{Definitions and theorem used}
\label{deftab}
\end{longtable}
\section{\textbf{Solution}}
\renewcommand{\thetable}{2}
\begin{longtable}{|l|l|}
\hline
\endhead
\textbf{Statement}&\textbf{Observations}\\
\hline
Given&Function $\vec{F}:\mathbb{R}^n\times\mathbb{R}^n\rightarrow\mathbb{R}$, it is given as\\&\parbox{13cm}{\begin{align}
    \vec{F}(\vec{x},\vec{y})=\langle\vec{Ax},\vec{y}\rangle=\vec{x}^T\vec{A}^T\vec{y}\label{F}
\end{align}}\\&where $\vec{x}$,$\vec{y}\in\mathbb{R}^n$\\&Using property \eqref{prop1}, we can also get\\&\parbox{13cm}{\begin{align}
    \implies\vec{F}(\vec{x},\vec{y})=\langle\vec{y},\vec{Ax}\rangle\\
    \implies\vec{F}(\vec{x},\vec{y})=\vec{y}^T\vec{A}\vec{x}\label{Fp}
\end{align}}\\
\hline
Total Derivative $D$&Now we will calculate $D\vec{F}(\vec{x},\vec{y})$\\&\parbox{13cm}{\begin{align}
    D\vec{F}(\vec{x},\vec{y})=\myvec{\frac{\partial \vec{F}}{\partial \vec{x}}&\frac{\partial \vec{F}}{\partial \vec{y}}}\label{D}
\end{align}}\\&From \eqref{F},\eqref{Fp} we get\\&\parbox{13cm}{\begin{align}
    \frac{\partial \vec{F}}{\partial \vec{x}}=\vec{y}^T\vec{A}\label{df1}\\
    \frac{\partial \vec{F}}{\partial \vec{y}}=\vec{x}^T\vec{A}^T\label{df2}
\end{align}}\\&Substitute \eqref{df1} and \eqref{df2} in \eqref{D}\\&\parbox{13cm}{\begin{align}
    D\vec{F}(\vec{x},\vec{y})=\myvec{\vec{y}^T\vec{A}&\vec{x}^T\vec{A}^T}_{1\times n^2}\label{Dsol}
\end{align}}\\
\hline
\caption{Observations}
\label{obs}
\end{longtable}
\renewcommand{\thetable}{3}
\begin{longtable}{|l|l|l|}
\hline
\endhead
\textbf{Option}&\textbf{Solution}&\textbf{True/}\\&&\textbf{False}\\
\hline
1&First we calculate $(D\vec{F}(\vec{x},\vec{y}))(\vec{u},\vec{v})$ where $\vec{u}$,$\vec{v}\in\mathbb{R}^n$&\\&Using \eqref{Dsol}and block matrix multiplication we get&\\&\parbox{14cm}{\begin{align}
    (D\vec{F}(\vec{x},\vec{y}))(\vec{u},\vec{v})=\myvec{\vec{y}^T\vec{A}&\vec{x}^T\vec{A}^T}\myvec{\vec{u}\\\vec{v}}\\
    \implies(D\vec{F}(\vec{x},\vec{y}))(\vec{u},\vec{v})=\vec{y}^T\vec{A}\vec{u}+\vec{x}^T\vec{A}^T\vec{v}\label{eq1}\\
    (D\vec{F}(\vec{x},\vec{y}))(\vec{u},\vec{v})=\langle\vec{y},\vec{Au}\rangle+\langle\vec{Ax},\vec{v}\rangle
\end{align}}&\\&Using property \eqref{prop1} we get&\\&\parbox{14cm}{\begin{align}
    (D\vec{F}(\vec{x},\vec{y}))(\vec{u},\vec{v})=\langle\vec{Au},\vec{y}\rangle+\langle\vec{Ax},\vec{v}\rangle\label{p1}
\end{align}}&\\
\hline
2.&Using \eqref{eq1}, if $\vec{u}=0$ and $\vec{v}=0$ then we get&\\&\parbox{14cm}{\begin{align}
    (D\vec{F}(\vec{x},\vec{y}))(0,0)=0\label{p2}
\end{align}}&True\\
\hline
3.&Since from \eqref{Dsol} we can say that $D\vec{F}(\vec{x},\vec{y})$ will exist for any $(\vec{x},\vec{y})\in\mathbb{R}^n\times\mathbb{R}^n$.&False\\&&\\
\hline
4.&From \eqref{Dsol}, if $(\vec{x},\vec{y})=(0,0)$ we get&\\&\parbox{14cm}{\begin{align}
    D\vec{F}(\vec{x},\vec{y})|_{(0,0)}=0
\end{align}}&\\&Therefore we can say that $D\vec{F}(\vec{x},\vec{y})$ will exist at $(\vec{x},\vec{y})=(0,0)$.&False\\
\hline
\caption{Solution}
\label{sol}
\end{longtable}
\end{document}
