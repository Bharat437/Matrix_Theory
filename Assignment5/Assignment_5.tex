\documentclass[journal,12pt,twocolumn]{IEEEtran}

\usepackage{setspace}
\usepackage{gensymb}
\usepackage{standalone}
\singlespacing


\usepackage[cmex10]{amsmath}

\usepackage{amsthm}
\usepackage{mathrsfs}
\usepackage{txfonts}
\usepackage{stfloats}
\usepackage{bm}
\usepackage{cite}
\usepackage{cases}
\usepackage{subfig}

\usepackage{longtable}
\usepackage{multirow}

\usepackage{enumitem}
\usepackage{mathtools}
\usepackage{steinmetz}
\usepackage{tikz}
\usepackage{circuitikz}
\usepackage{verbatim}
\usepackage{tfrupee}
\usepackage[breaklinks=true]{hyperref}
\usepackage{graphicx}
\usepackage{tkz-euclide}

\usetikzlibrary{calc,math}
\usepackage{listings}
    \usepackage{color}                                            %%
    \usepackage{array}                                            %%
    \usepackage{longtable}                                        %%
    \usepackage{calc}                                             %%
    \usepackage{multirow}                                         %%
    \usepackage{hhline}                                           %%
    \usepackage{ifthen}                                           %%
    \usepackage{lscape}     
\usepackage{multicol}
\usepackage{chngcntr}

\DeclareMathOperator*{\Res}{Res}

\renewcommand\thesection{\arabic{section}}
\renewcommand\thesubsection{\thesection.\arabic{subsection}}
\renewcommand\thesubsubsection{\thesubsection.\arabic{subsubsection}}

\renewcommand\thesectiondis{\arabic{section}}
\renewcommand\thesubsectiondis{\thesectiondis.\arabic{subsection}}
\renewcommand\thesubsubsectiondis{\thesubsectiondis.\arabic{subsubsection}}


\hyphenation{op-tical net-works semi-conduc-tor}
\def\inputGnumericTable{}                                 %%

\lstset{
%language=C,
frame=single, 
breaklines=true,
columns=fullflexible
}
\usepackage{pdfpages}
\usepackage{pgfplots}

\begin{document}


\newtheorem{theorem}{Theorem}[section]
\newtheorem{problem}{Problem}
\newtheorem{proposition}{Proposition}[section]
\newtheorem{lemma}{Lemma}[section]
\newtheorem{corollary}[theorem]{Corollary}
\newtheorem{example}{Example}[section]
\newtheorem{definition}[problem]{Definition}

\newcommand{\BEQA}{\begin{eqnarray}}
\newcommand{\EEQA}{\end{eqnarray}}
\newcommand{\define}{\stackrel{\triangle}{=}}
\bibliographystyle{IEEEtran}
\providecommand{\mbf}{\mathbf}
\providecommand{\pr}[1]{\ensuremath{\Pr\left(#1\right)}}
\providecommand{\qfunc}[1]{\ensuremath{Q\left(#1\right)}}
\providecommand{\sbrak}[1]{\ensuremath{{}\left[#1\right]}}
\providecommand{\lsbrak}[1]{\ensuremath{{}\left[#1\right.}}
\providecommand{\rsbrak}[1]{\ensuremath{{}\left.#1\right]}}
\providecommand{\brak}[1]{\ensuremath{\left(#1\right)}}
\providecommand{\lbrak}[1]{\ensuremath{\left(#1\right.}}
\providecommand{\rbrak}[1]{\ensuremath{\left.#1\right)}}
\providecommand{\cbrak}[1]{\ensuremath{\left\{#1\right\}}}
\providecommand{\lcbrak}[1]{\ensuremath{\left\{#1\right.}}
\providecommand{\rcbrak}[1]{\ensuremath{\left.#1\right\}}}
\theoremstyle{remark}
\newtheorem{rem}{Remark}
\newcommand{\sgn}{\mathop{\mathrm{sgn}}}
\providecommand{\abs}[1]{\left\vert#1\right\vert}
\providecommand{\res}[1]{\Res\displaylimits_{#1}} 
\providecommand{\norm}[1]{\left\lVert#1\right\rVert}
%\providecommand{\norm}[1]{\lVert#1\rVert}
\providecommand{\mtx}[1]{\mathbf{#1}}
\providecommand{\mean}[1]{E\left[ #1 \right]}
\providecommand{\fourier}{\overset{\mathcal{F}}{ \rightleftharpoons}}
%\providecommand{\hilbert}{\overset{\mathcal{H}}{ \rightleftharpoons}}
\providecommand{\system}{\overset{\mathcal{H}}{ \longleftrightarrow}}
	%\newcommand{\solution}[2]{\textbf{Solution:}{#1}}
\newcommand{\solution}{\noindent \textbf{Solution: }}
\newcommand{\cosec}{\,\text{cosec}\,}
\providecommand{\dec}[2]{\ensuremath{\overset{#1}{\underset{#2}{\gtrless}}}}
\newcommand{\myvec}[1]{\ensuremath{\begin{pmatrix}#1\end{pmatrix}}}
\newcommand{\mydet}[1]{\ensuremath{\begin{vmatrix}#1\end{vmatrix}}}
\numberwithin{equation}{subsection}
\makeatletter
\@addtoreset{figure}{problem}
\makeatother
\let\StandardTheFigure\thefigure
\let\vec\mathbf
\renewcommand{\thefigure}{\theproblem}
\def\putbox#1#2#3{\makebox[0in][l]{\makebox[#1][l]{}\raisebox{\baselineskip}[0in][0in]{\raisebox{#2}[0in][0in]{#3}}}}
     \def\rightbox#1{\makebox[0in][r]{#1}}
     \def\centbox#1{\makebox[0in]{#1}}
     \def\topbox#1{\raisebox{-\baselineskip}[0in][0in]{#1}}
     \def\midbox#1{\raisebox{-0.5\baselineskip}[0in][0in]{#1}}
\vspace{3cm}
\title{Assignment 5}
\author{AVVARU BHARAT}
\maketitle
\newpage
\bigskip
\renewcommand{\thefigure}{\theenumi}
\renewcommand{\thetable}{\theenumi}
Download latex-tikz codes from 
%
\begin{lstlisting}
https://github.com/Bharat437/Matrix_Theory/tree/master/Assignment5
\end{lstlisting}
\section{\textbf{Question}}
(loney 13.8) Q. Find the value of k so that the following equation may represent pair of straight lines: $12x^2+kxy+2y^2+11x-5y+2=0$.
\section{\textbf{Explanation}}

Comparing the given equation with the general equation of second degree given as below:
\begin{align}
    ax^2+2bxy+cy^2++2dx+2ey+f=0
\end{align}

we will get $a=12$, $b=\frac{k}{2}$, $c=2$, $d=\frac{11}{2}$, $e=-\frac{5}{2}$, $f=2$.

The general equation can be expressed as:
\begin{align}
    \vec{x}^T\vec{V}\vec{x}+2\vec{u}^T\vec{x}+f=0
\end{align}
where
\begin{align}
    \vec{V}=\vec{V}^T=\myvec{a & b \\ b & c}=\myvec{12 & \frac{k}{2} \\ \frac{k}{2} & 2}\\
    \vec{u}=\myvec{d \\ e}=\myvec{\frac{11}{2} \\ -\frac{5}{2}}
\end{align}

The given equation represents pair of straight lines if
\begin{align}
    \mydet{\vec{V} & \vec{u} \\ \vec{u}^T & f}=0\\
    \implies\label{eq:6}\mydet{12 & \frac{k}{2} & \frac{11}{2} \\ \frac{k}{2} & 2 & -\frac{5}{2} \\ \frac{11}{2} & -\frac{5}{2} & 2}=0
\end{align}

The matrix in \eqref{eq:6} must be singular matrix and in echelon form of the matrix should consist a row with all zeros.
\begin{align}
    \implies\myvec{12 & \frac{k}{2} & \frac{11}{2} \\ \frac{k}{2} & 2 & -\frac{5}{2} \\ \frac{11}{2} & -\frac{5}{2} & 2}\\
    \implies\myvec{24 & k & 11 \\ k & 4 & -5 \\ 11 & -5 & 4}\\
    \xleftrightarrow[]{R_2\leftarrow 24R_2-kR1}\myvec{24 & k & 11 \\ 0 & 96-k^2 & -120-11k \\ 11 & -5 & 4}\\
    \xleftrightarrow[]{R_3\leftarrow 24R_3-11R1}\myvec{24 & k & 11 \\ 0 & 96-k^2 & -120-11k \\ 0 & -120-11k & -25}
\end{align}
\begin{multline}
    \xleftrightarrow[]{R_3\leftarrow (96-k^2)R_3-(-120-11k)R2}\\\myvec{24 & k & 11 \\ 0 & 96-k^2 & -120-11k \\ 0 & 0 & -96k^2-2640k-16800}\label{eq:11}
\end{multline}

In \eqref{eq:11}, the elements in last row must consist all zeros. For this to happen we should find k value.
\begin{align}
    \implies-96k^2-2640k-16800=0\\
    \implies2k^2+55k+350=0\\
    \implies(10+k)(2k+35)=0\\
    \implies k=-10 \text{ and } k=-\frac{35}{2}
\end{align}
Therefore, for $k=-10$ and $k=-\frac{35}{2}$ the given equation represents pair of straight lines.
\end{document}
