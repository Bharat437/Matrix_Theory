\documentclass[journal,12pt]{IEEEtran}
\usepackage{longtable}
\usepackage{setspace}
\usepackage{gensymb}
\singlespacing
\usepackage[cmex10]{amsmath}
\newcommand\myemptypage{
	\null
	\thispagestyle{empty}
	\addtocounter{page}{-1}
	\newpage
}
\usepackage{amsthm}
\usepackage{mdframed}
\usepackage{mathrsfs}
\usepackage{txfonts}
\usepackage{stfloats}
\usepackage{bm}
\usepackage{cite}
\usepackage{cases}
\usepackage{subfig}

\usepackage{longtable}
\usepackage{multirow}

\usepackage{enumitem}
\usepackage{mathtools}
\usepackage{steinmetz}
\usepackage{tikz}
\usepackage{circuitikz}
\usepackage{verbatim}
\usepackage{tfrupee}
\usepackage[breaklinks=true]{hyperref}
\usepackage{graphicx}
\usepackage{tkz-euclide}

\usetikzlibrary{calc,math}
\usepackage{listings}
    \usepackage{color}                                            %%
    \usepackage{array}                                            %%
    \usepackage{longtable}                                        %%
    \usepackage{calc}                                             %%
    \usepackage{multirow}                                         %%
    \usepackage{hhline}                                           %%
    \usepackage{ifthen}                                           %%
    \usepackage{lscape}     
\usepackage{multicol}
\usepackage{chngcntr}

\DeclareMathOperator*{\Res}{Res}

\renewcommand\thesection{\arabic{section}}
\renewcommand\thesubsection{\thesection.\arabic{subsection}}
\renewcommand\thesubsubsection{\thesubsection.\arabic{subsubsection}}

\renewcommand\thesectiondis{\arabic{section}}
\renewcommand\thesubsectiondis{\thesectiondis.\arabic{subsection}}
\renewcommand\thesubsubsectiondis{\thesubsectiondis.\arabic{subsubsection}}


\hyphenation{op-tical net-works semi-conduc-tor}
\def\inputGnumericTable{}                                 %%

\lstset{
%language=C,
frame=single, 
breaklines=true,
columns=fullflexible
}
\begin{document}
\onecolumn

\newtheorem{theorem}{Theorem}[section]
\newtheorem{problem}{Problem}
\newtheorem{proposition}{Proposition}[section]
\newtheorem{lemma}{Lemma}[section]
\newtheorem{corollary}[theorem]{Corollary}
\newtheorem{example}{Example}[section]
\newtheorem{definition}[problem]{Definition}

\newcommand{\BEQA}{\begin{eqnarray}}
\newcommand{\EEQA}{\end{eqnarray}}
\newcommand{\define}{\stackrel{\triangle}{=}}
\bibliographystyle{IEEEtran}
\raggedbottom
\setlength{\parindent}{0pt}
\providecommand{\mbf}{\mathbf}
\providecommand{\pr}[1]{\ensuremath{\Pr\left(#1\right)}}
\providecommand{\qfunc}[1]{\ensuremath{Q\left(#1\right)}}
\providecommand{\sbrak}[1]{\ensuremath{{}\left[#1\right]}}
\providecommand{\lsbrak}[1]{\ensuremath{{}\left[#1\right.}}
\providecommand{\rsbrak}[1]{\ensuremath{{}\left.#1\right]}}
\providecommand{\brak}[1]{\ensuremath{\left(#1\right)}}
\providecommand{\lbrak}[1]{\ensuremath{\left(#1\right.}}
\providecommand{\rbrak}[1]{\ensuremath{\left.#1\right)}}
\providecommand{\cbrak}[1]{\ensuremath{\left\{#1\right\}}}
\providecommand{\lcbrak}[1]{\ensuremath{\left\{#1\right.}}
\providecommand{\rcbrak}[1]{\ensuremath{\left.#1\right\}}}
\theoremstyle{remark}
\newtheorem{rem}{Remark}
\newcommand{\sgn}{\mathop{\mathrm{sgn}}}
\providecommand{\abs}[1]{\left\vert#1\right\vert}
\providecommand{\res}[1]{\Res\displaylimits_{#1}} 
\providecommand{\norm}[1]{\left\lVert#1\right\rVert}
%\providecommand{\norm}[1]{\lVert#1\rVert}
\providecommand{\mtx}[1]{\mathbf{#1}}
\providecommand{\mean}[1]{E\left[ #1 \right]}
\providecommand{\fourier}{\overset{\mathcal{F}}{ \rightleftharpoons}}
%\providecommand{\hilbert}{\overset{\mathcal{H}}{ \rightleftharpoons}}
\providecommand{\system}{\overset{\mathcal{H}}{ \longleftrightarrow}}
	%\newcommand{\solution}[2]{\textbf{Solution:}{#1}}
\newcommand{\solution}{\noindent \textbf{Solution: }}
\newcommand{\cosec}{\,\text{cosec}\,}
\providecommand{\dec}[2]{\ensuremath{\overset{#1}{\underset{#2}{\gtrless}}}}
\newcommand{\myvec}[1]{\ensuremath{\begin{pmatrix}#1\end{pmatrix}}}
\newcommand{\mydet}[1]{\ensuremath{\begin{vmatrix}#1\end{vmatrix}}}
\numberwithin{equation}{subsection}
\makeatletter
\@addtoreset{figure}{problem}
\makeatother
\let\StandardTheFigure\thefigure
\let\vec\mathbf
\renewcommand{\thefigure}{\theproblem}
\def\putbox#1#2#3{\makebox[0in][l]{\makebox[#1][l]{}\raisebox{\baselineskip}[0in][0in]{\raisebox{#2}[0in][0in]{#3}}}}
     \def\rightbox#1{\makebox[0in][r]{#1}}
     \def\centbox#1{\makebox[0in]{#1}}
     \def\topbox#1{\raisebox{-\baselineskip}[0in][0in]{#1}}
     \def\midbox#1{\raisebox{-0.5\baselineskip}[0in][0in]{#1}}
\vspace{3cm}
\title{Assignment 13}
\author{AVVARU BHARAT - EE20MTECH11008}
\maketitle
\bigskip
\renewcommand{\thefigure}{\theenumi}
\renewcommand{\thetable}{\theenumi}
%
Download the latex-tikz codes from 
%
\begin{lstlisting}
https://github.com/Bharat437/Matrix_Theory/tree/master/Assignment13
\end{lstlisting}
\section{\textbf{Problem}}
(UGC,Dec 2018,77) : \\
%
Define a real values function $\vec{B}$ on $\mathbb{R}^2\times\mathbb{R}^2$ as follows. If $v=(x_1,x_2)$, $w=(y_1,y_2)$ belong to $\mathbb{R}^2$ define $\vec{B}(u,w)=x_1y_1-x_1y_2-x_2y_1+4x_2y_2$. Let $v_0=(1,0)$ and let $\vec{W}=\cbrak{v\in\mathbb{R}^2:\vec{B}(v_0,v)=0}$. Then $\vec{W}$
\begin{enumerate}
    \item is not a subspace of $\mathbb{R}^2$
    \item equals $\cbrak{(0,0)}$
    \item is the y axis
    \item is the line passing through (0,0) and (1,1)
\end{enumerate}
\section{\textbf{Explanation}}
\renewcommand{\thetable}{1}
\begin{longtable}{|l|l|}
\hline
\endhead
\textbf{Subspace}&A non-empty subset $\vec{W}$ of $\vec{V}$ is a subspace of $\vec{V}$ if and only if for each pair of vectors $\vec{\alpha}$,\\& $\vec{\beta}$ in $\vec{W}$ and each scalar $c$ in $\vec{F}$ the vector $c\vec{\alpha}+\vec{\beta}$ is again in $\vec{W}$.\\
\hline
\caption{Definitions and theorem used}
\label{deftab}
\end{longtable}
\section{\textbf{Solution}}
\renewcommand{\thetable}{2}
\begin{longtable}{|l|l|}
\hline
\endhead
\textbf{Statement}&\textbf{Observations}\\
\hline
Given&\parbox{10cm}{\begin{align}
    \vec{W}&=\cbrak{v\in\mathbb{R}^2:\vec{B}(v_0,v)=0}\label{W}\\
    v_0&=(1,0)\label{v0}\\
    \vec{B}(u,w)&=x_1y_1-x_1y_2-x_2y_1+4x_2y_2\label{B}
\end{align}}\\&From \eqref{v0} and \eqref{B}, we will calculate $\vec{B}(v_0,v)$\\&\parbox{10cm}{\begin{align}
    \vec{B}(v_0,v)=y_1-y_2\label{Bv0}
\end{align}}\\&$\vec{B}(v_0,v)=0$ if and only if $y_1=y_2$\\&Therefore, $\vec{W}$ consists points which have same x and y\\&coordinates.\\
\hline
\caption{Observations}
\label{obs}
\end{longtable}
\renewcommand{\thetable}{3}
\begin{longtable}{|l|l|l|}
\hline
\endhead
\textbf{Option}&\textbf{Solution}&\textbf{True/False}\\
\hline
1.&Now we will see whether $\vec{W}$ is a subspace or not.&\\&Let $\vec{\alpha}=(m,m)$ and $\vec{\beta}=(n,n)$ be two pair of vectors in $\vec{W}$ where $\vec{\alpha}$, $\vec{\beta}\in\mathbb{R}^2$&\\& and c be a scalar value in $\mathbb{R}$.&\\&Now we will see whether the vector $c\vec{\alpha}+\vec{\beta}$ is in $\vec{W}$ or not.&\\
&Here&\\&\parbox{13cm}{\begin{align}
    c\vec{\alpha}+\vec{\beta}=(cm+n,cm+n)
\end{align}}&\\&Now we will calculate $\vec{B}(v_0,c\vec{\alpha}+\vec{\beta})$ using \eqref{Bv0}&False\\&\parbox{13cm}{\begin{align}
    \implies\vec{B}(v_0,c\vec{\alpha}+\vec{\beta})=(cm+n)-(cm+n)\\
    \implies\vec{B}(v_0,c\vec{\alpha}+\vec{\beta})=0\label{p1}
\end{align}}&\\&From \eqref{p1}, we can say that vector $c\vec{\alpha}+\vec{\beta}\in\vec{W}$.&\\&Therefore, $\vec{W}$ is a subspace of $\mathbb{R}$&\\
\hline
2.&From Table \ref{obs}, we got $\vec{W}$ consists points which have same x and y &\\&coordinates.&\\&For example vector $\vec{u}=(1,1)\in\mathbb{R}^2$, we will calculate $\vec{B}(v_0,u)$&\\&\parbox{13cm}{\begin{align}
    \implies\vec{B}(v_0,u)=1-1=0\label{p2}
\end{align}}&False\\&From \eqref{p2}, we can say that vector $\vec{u}\in\vec{W}$.&\\&Therefore, $\vec{W}\neq\cbrak{(0,0)}$&\\
\hline
3.&Let us consider a point on y-axis, $p=(3,0)$ we will calculate $\vec{B}(v_0,p)$&\\&\parbox{13cm}{\begin{align}
    \implies\vec{B}(v_0,p)=3-0=3\\
    \implies\vec{B}(v_0,p)\neq0\label{p3}
\end{align}}&False\\&From \eqref{p3}, we can say that vector $\vec{p}\not\in\vec{W}$.&\\&Therefore, all points in $\vec{W}$ are not on y-axis.&\\
\hline
4.&The direction vector $\vec{m}$ and normal vector $\vec{n}$ of the line through $\vec{M}=(0,0)$&\\& and $\vec{N}=(1,1)$ is&\\&\parbox{13cm}{\begin{align}
    \vec{m}=\myvec{1\\1}-\myvec{0\\0}=\myvec{1\\1}\\
    \vec{n}=\myvec{0&-1\\1&0}\vec{m}=\myvec{0&-1\\1&0}\myvec{1\\1}\\
    \implies\vec{n}=\myvec{-1\\1}
\end{align}}&\\&The equation of line can be obtained as&True\\&\parbox{13cm}{\begin{align}
    \vec{n}^T(\vec{x}-\vec{M})=0\\
    \implies\myvec{-1&1}\left(\vec{x}-\myvec{0\\0}\right)=0\\
    \implies\myvec{-1&1}\vec{x}=0\label{leq}
\end{align}}&\\&\eqref{leq} is the equation of line. Therefore, We can say that the line passes&\\& through the points which is having same x and y coordinates.&\\&Therefore From Table \ref{obs}, all points in $\vec{W}$ are on the line passing through $\vec{M}$&\\& and $\vec{N}$&\\
\hline
\caption{Solution}
\label{sol}
\end{longtable}
\end{document}
