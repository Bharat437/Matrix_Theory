\documentclass[journal,12pt,twocolumn]{IEEEtran}

\usepackage{setspace}
\usepackage{gensymb}
\usepackage{enumitem}
\singlespacing


\usepackage[cmex10]{amsmath}

\usepackage{amsthm}

\usepackage{mathrsfs}
\usepackage{txfonts}
\usepackage{stfloats}
\usepackage{bm}
\usepackage{cite}
\usepackage{cases}
\usepackage{subfig}

\usepackage{longtable}
\usepackage{multirow}

\usepackage{enumitem}
\usepackage{mathtools}
\usepackage{steinmetz}
\usepackage{tikz}
\usepackage{circuitikz}
\usepackage{verbatim}
\usepackage{tfrupee}
\usepackage[breaklinks=true]{hyperref}

\usepackage{tkz-euclide}

\usetikzlibrary{calc,math}
\usepackage{listings}
    \usepackage{color}                                            %%
    \usepackage{array}                                            %%
    \usepackage{longtable}                                        %%
    \usepackage{calc}                                             %%
    \usepackage{multirow}                                         %%
    \usepackage{hhline}                                           %%
    \usepackage{ifthen}                                           %%
    \usepackage{lscape}     
\usepackage{multicol}
\usepackage{chngcntr}

\DeclareMathOperator*{\Res}{Res}

\renewcommand\thesection{\arabic{section}}
\renewcommand\thesubsection{\thesection.\arabic{subsection}}
\renewcommand\thesubsubsection{\thesubsection.\arabic{subsubsection}}

\renewcommand\thesectiondis{\arabic{section}}
\renewcommand\thesubsectiondis{\thesectiondis.\arabic{subsection}}
\renewcommand\thesubsubsectiondis{\thesubsectiondis.\arabic{subsubsection}}


\hyphenation{op-tical net-works semi-conduc-tor}
\def\inputGnumericTable{}                                 %%

\lstset{
%language=C,
frame=single, 
breaklines=true,
columns=fullflexible
}
\begin{document}


\newtheorem{theorem}{Theorem}[section]
\newtheorem{problem}{Problem}
\newtheorem{proposition}{Proposition}[section]
\newtheorem{lemma}{Lemma}[section]
\newtheorem{corollary}[theorem]{Corollary}
\newtheorem{example}{Example}[section]
\newtheorem{definition}[problem]{Definition}

\newcommand{\BEQA}{\begin{eqnarray}}
\newcommand{\EEQA}{\end{eqnarray}}
\newcommand{\define}{\stackrel{\triangle}{=}}
\bibliographystyle{IEEEtran}
\providecommand{\mbf}{\mathbf}
\providecommand{\pr}[1]{\ensuremath{\Pr\left(#1\right)}}
\providecommand{\qfunc}[1]{\ensuremath{Q\left(#1\right)}}
\providecommand{\sbrak}[1]{\ensuremath{{}\left[#1\right]}}
\providecommand{\lsbrak}[1]{\ensuremath{{}\left[#1\right.}}
\providecommand{\rsbrak}[1]{\ensuremath{{}\left.#1\right]}}
\providecommand{\brak}[1]{\ensuremath{\left(#1\right)}}
\providecommand{\lbrak}[1]{\ensuremath{\left(#1\right.}}
\providecommand{\rbrak}[1]{\ensuremath{\left.#1\right)}}
\providecommand{\cbrak}[1]{\ensuremath{\left\{#1\right\}}}
\providecommand{\lcbrak}[1]{\ensuremath{\left\{#1\right.}}
\providecommand{\rcbrak}[1]{\ensuremath{\left.#1\right\}}}
\theoremstyle{remark}
\newtheorem{rem}{Remark}
\newcommand{\sgn}{\mathop{\mathrm{sgn}}}
\providecommand{\abs}[1]{\left\vert#1\right\vert}
\providecommand{\res}[1]{\Res\displaylimits_{#1}} 
\providecommand{\norm}[1]{\left\lVert#1\right\rVert}
%\providecommand{\norm}[1]{\lVert#1\rVert}
\providecommand{\mtx}[1]{\mathbf{#1}}
\providecommand{\mean}[1]{E\left[ #1 \right]}
\providecommand{\fourier}{\overset{\mathcal{F}}{ \rightleftharpoons}}
%\providecommand{\hilbert}{\overset{\mathcal{H}}{ \rightleftharpoons}}
\providecommand{\system}{\overset{\mathcal{H}}{ \longleftrightarrow}}
	%\newcommand{\solution}[2]{\textbf{Solution:}{#1}}
\newcommand{\solution}{\noindent \textbf{Solution: }}
\newcommand{\cosec}{\,\text{cosec}\,}
\providecommand{\dec}[2]{\ensuremath{\overset{#1}{\underset{#2}{\gtrless}}}}
\newcommand{\myvec}[1]{\ensuremath{\begin{pmatrix}#1\end{pmatrix}}}
\newcommand{\mydet}[1]{\ensuremath{\begin{vmatrix}#1\end{vmatrix}}}
\numberwithin{equation}{subsection}
\makeatletter
\@addtoreset{figure}{problem}
\makeatother
\let\StandardTheFigure\thefigure
\let\vec\mathbf
\renewcommand{\thefigure}{\theproblem}
\def\putbox#1#2#3{\makebox[0in][l]{\makebox[#1][l]{}\raisebox{\baselineskip}[0in][0in]{\raisebox{#2}[0in][0in]{#3}}}}
     \def\rightbox#1{\makebox[0in][r]{#1}}
     \def\centbox#1{\makebox[0in]{#1}}
     \def\topbox#1{\raisebox{-\baselineskip}[0in][0in]{#1}}
     \def\midbox#1{\raisebox{-0.5\baselineskip}[0in][0in]{#1}}
\vspace{3cm}
\title{Assignment 10}
\author{AVVARU BHARAT}
\maketitle
\newpage
\bigskip
\renewcommand{\thefigure}{\theenumi}
\renewcommand{\thetable}{\theenumi}
Download latex-tikz codes from 
%
\begin{lstlisting}
https://github.com/Bharat437/Matrix_Theory/tree/master/Assignment10
\end{lstlisting}
%
\section{Problem}
Let $\vec{V}$ be the vector space over the complex numbers of all functions from $\mathbb{R}$ into $\mathbb{C}$, i.e., the space of all complex-valued functions on the real line. Let $f_1(x)=1$, $f_2(x)=e^{ix}$, $f_3(x)=e^{-ix}$.

(1) Prove that $f_1$, $f_2$, and $f_3$ are linearly independent.

(2) Let $g_1(x)=1$, $g_2(x)=\cos{x}$, $g_3(x)=\sin{x}$. Find an invertible $3\times3$ matrix $\vec{P}$ such that
\begin{align}
    g_j=\sum\limits_{i=1}^3 \vec{P}_{ij}f_i\label{eq}
\end{align}
\section{Solution}
\begin{enumerate}
\item
Given,
\begin{align}
    f_1(x)=1\label{f1}\\
    f_2(x)=e^{ix}\label{f2}\\
    f_3(x)=e^{-ix}\label{f3}
\end{align}

For $f_1$, $f_2$, and $f_3$ to be linearly independent, the following condition must satisfy.
\begin{align}
    k_1f_1+k_2f_2+k_3f_3=0\label{indeq}
\end{align}
$\forall k_i=0$ and $i=$1,2,3

Substitute \eqref{f1},\eqref{f2}, \eqref{f3} in \eqref{indeq}, we get
\begin{align}
    k_1+k_2e^{ix}+k_3e^{-ix}=0\label{indeq1}
\end{align}
We know from Euler formula,
\begin{align} 
e^{ix}=\cos{x}+i\sin{x}\label{exp} 
\end{align}
Substitute \eqref{exp} in \eqref{indeq1}, we get
\begin{align}
k_1+k_2\cos{x}+ik_2\sin{x}+k_3\cos{x}-ik_3\sin{x}=0\label{subind}
\end{align}
Now equate real and imaginary parts of \eqref{subind}, we get
\begin{align}
k_1+k_2\cos{x}+k_3\cos{x}=0\label{real}\\
k_2\sin{x}-k_3\sin{x}=0\label{imag}\\
\implies k_2=k_3\label{step1}
\end{align}
Substitute \eqref{step1} in \eqref{real}, we get
\begin{align}
\implies k_1+2k_3\cos{x}=0\label{step2}
\end{align}
Differentiating \eqref{step2} with respect to x, we get
\begin{align}
\implies -2k_3\sin{x}=0\\
\implies k_3=0\label{k3}
\end{align}
Substitute \eqref{k3} in \eqref{step1} and \eqref{step2}, we get 
\begin{align}
    k_3=k_2=0\label{k3k2}\\
    k_1=0\label{k1}
\end{align}
Therefore, from \eqref{k3k2} and \eqref{k1}, we can say that
\begin{align}
    k_1f_1+k_2f_2+k_3f_3=0
\end{align}
$\forall k_i=0$ and $i=$1,2,3\\
Hence, $f_1, f_2, and f_3$ are linearly independent
\item
Given,
\begin{align}
    g_1(x)&=1=f_1\label{g1}\\
    g_2(x)&=\cos{x}=\frac{e^{ix}+e{-ix}}{2}=\frac{f_2}{2}+\frac{f_3}{2}\label{g2}
\end{align}
\begin{multline}
    g_3(x)=\sin{x}=\frac{e^{ix}-e{-ix}}{2i}=\frac{f_2}{2i}-\frac{f_3}{2i}\\=-\frac{i}{2}f_2+\frac{i}{2}f_3\label{g3}
\end{multline}

Now \eqref{g1}, \eqref{g2}, \eqref{g3} can be converted to matrix form as below.
\begin{align}
    \myvec{g_1&g_2&g_3}=\myvec{f_1&f_2&f_3}\myvec{1&0&0\\0&\frac{1}{2}&-\frac{i}{2}\\0&\frac{1}{2}&\frac{i}{2}}
\end{align}

Therefore, on comparing with \eqref{eq} we get
\begin{align}
    \vec{P}=\myvec{1&0&0\\0&\frac{1}{2}&-\frac{i}{2}\\0&\frac{1}{2}&\frac{i}{2}}
\end{align}

Now we will verify $\Vec{P}$ is invertible or not by row reduction.
\begin{align}
    \myvec{1&0&0\\0&\frac{1}{2}&-\frac{i}{2}\\0&\frac{1}{2}&\frac{i}{2}}\xleftrightarrow{R_3=R_3-R_2}\myvec{1&0&0\\0&\frac{1}{2}&-\frac{i}{2}\\0&0&i}
\end{align}

we got rank of matrix $\vec{P}$ is 3 and it is full rank matrix. Therefore, $\vec{P}$ is invertible matrix.

Hence verified it.
\end{enumerate}
\end{document}
