\documentclass[journal,12pt]{IEEEtran}
\usepackage{longtable}
\usepackage{setspace}
\usepackage{gensymb}
\singlespacing
\usepackage[cmex10]{amsmath}
\newcommand\myemptypage{
	\null
	\thispagestyle{empty}
	\addtocounter{page}{-1}
	\newpage
}
\usepackage{amsthm}
\usepackage{mdframed}
\usepackage{mathrsfs}
\usepackage{txfonts}
\usepackage{stfloats}
\usepackage{bm}
\usepackage{cite}
\usepackage{cases}
\usepackage{subfig}

\usepackage{longtable}
\usepackage{multirow}

\usepackage{enumitem}
\usepackage{mathtools}
\usepackage{steinmetz}
\usepackage{tikz}
\usepackage{circuitikz}
\usepackage{verbatim}
\usepackage{tfrupee}
\usepackage[breaklinks=true]{hyperref}
\usepackage{graphicx}
\usepackage{tkz-euclide}

\usetikzlibrary{calc,math}
\usepackage{listings}
    \usepackage{color}                                            %%
    \usepackage{array}                                            %%
    \usepackage{longtable}                                        %%
    \usepackage{calc}                                             %%
    \usepackage{multirow}                                         %%
    \usepackage{hhline}                                           %%
    \usepackage{ifthen}                                           %%
    \usepackage{lscape}     
\usepackage{multicol}
\usepackage{chngcntr}

\DeclareMathOperator*{\Res}{Res}

\renewcommand\thesection{\arabic{section}}
\renewcommand\thesubsection{\thesection.\arabic{subsection}}
\renewcommand\thesubsubsection{\thesubsection.\arabic{subsubsection}}

\renewcommand\thesectiondis{\arabic{section}}
\renewcommand\thesubsectiondis{\thesectiondis.\arabic{subsection}}
\renewcommand\thesubsubsectiondis{\thesubsectiondis.\arabic{subsubsection}}


\hyphenation{op-tical net-works semi-conduc-tor}
\def\inputGnumericTable{}                                 %%

\lstset{
%language=C,
frame=single, 
breaklines=true,
columns=fullflexible
}
\begin{document}
\onecolumn

\newtheorem{theorem}{Theorem}[section]
\newtheorem{problem}{Problem}
\newtheorem{proposition}{Proposition}[section]
\newtheorem{lemma}{Lemma}[section]
\newtheorem{corollary}[theorem]{Corollary}
\newtheorem{example}{Example}[section]
\newtheorem{definition}[problem]{Definition}

\newcommand{\BEQA}{\begin{eqnarray}}
\newcommand{\EEQA}{\end{eqnarray}}
\newcommand{\define}{\stackrel{\triangle}{=}}
\bibliographystyle{IEEEtran}
\raggedbottom
\setlength{\parindent}{0pt}
\providecommand{\mbf}{\mathbf}
\providecommand{\pr}[1]{\ensuremath{\Pr\left(#1\right)}}
\providecommand{\qfunc}[1]{\ensuremath{Q\left(#1\right)}}
\providecommand{\sbrak}[1]{\ensuremath{{}\left[#1\right]}}
\providecommand{\lsbrak}[1]{\ensuremath{{}\left[#1\right.}}
\providecommand{\rsbrak}[1]{\ensuremath{{}\left.#1\right]}}
\providecommand{\brak}[1]{\ensuremath{\left(#1\right)}}
\providecommand{\lbrak}[1]{\ensuremath{\left(#1\right.}}
\providecommand{\rbrak}[1]{\ensuremath{\left.#1\right)}}
\providecommand{\cbrak}[1]{\ensuremath{\left\{#1\right\}}}
\providecommand{\lcbrak}[1]{\ensuremath{\left\{#1\right.}}
\providecommand{\rcbrak}[1]{\ensuremath{\left.#1\right\}}}
\theoremstyle{remark}
\newtheorem{rem}{Remark}
\newcommand{\sgn}{\mathop{\mathrm{sgn}}}
\providecommand{\abs}[1]{\left\vert#1\right\vert}
\providecommand{\res}[1]{\Res\displaylimits_{#1}} 
\providecommand{\norm}[1]{\left\lVert#1\right\rVert}
%\providecommand{\norm}[1]{\lVert#1\rVert}
\providecommand{\mtx}[1]{\mathbf{#1}}
\providecommand{\mean}[1]{E\left[ #1 \right]}
\providecommand{\fourier}{\overset{\mathcal{F}}{ \rightleftharpoons}}
%\providecommand{\hilbert}{\overset{\mathcal{H}}{ \rightleftharpoons}}
\providecommand{\system}{\overset{\mathcal{H}}{ \longleftrightarrow}}
	%\newcommand{\solution}[2]{\textbf{Solution:}{#1}}
\newcommand{\solution}{\noindent \textbf{Solution: }}
\newcommand{\cosec}{\,\text{cosec}\,}
\providecommand{\dec}[2]{\ensuremath{\overset{#1}{\underset{#2}{\gtrless}}}}
\newcommand{\myvec}[1]{\ensuremath{\begin{pmatrix}#1\end{pmatrix}}}
\newcommand{\mydet}[1]{\ensuremath{\begin{vmatrix}#1\end{vmatrix}}}
\numberwithin{equation}{subsection}
\makeatletter
\@addtoreset{figure}{problem}
\makeatother
\let\StandardTheFigure\thefigure
\let\vec\mathbf
\renewcommand{\thefigure}{\theproblem}
\def\putbox#1#2#3{\makebox[0in][l]{\makebox[#1][l]{}\raisebox{\baselineskip}[0in][0in]{\raisebox{#2}[0in][0in]{#3}}}}
     \def\rightbox#1{\makebox[0in][r]{#1}}
     \def\centbox#1{\makebox[0in]{#1}}
     \def\topbox#1{\raisebox{-\baselineskip}[0in][0in]{#1}}
     \def\midbox#1{\raisebox{-0.5\baselineskip}[0in][0in]{#1}}
\vspace{3cm}
\title{Assignment 14}
\author{AVVARU BHARAT - EE20MTECH11008}
\maketitle
\bigskip
\renewcommand{\thefigure}{\theenumi}
\renewcommand{\thetable}{\theenumi}
%
Download the latex-tikz codes from 
%
\begin{lstlisting}
https://github.com/Bharat437/Matrix_Theory/tree/master/Assignment14
\end{lstlisting}
\section{\textbf{Problem}}
(UGC,Dec 2015,74) : \\
%
Let $\vec{V}$ be a finite dimensional vector space over $\mathbb{R}$. Let $T:\vec{V}\rightarrow\vec{V}$ be a linear transformation such that $rank(\vec{T}^2)=rank(\vec{T})$. Then,
\begin{enumerate}
    \item $Kernel(\vec{T}^2)=Kernel(\vec{T})$
    \item $Range(\vec{T}^2)=Range(\vec{T})$
    \item $Kernel(\vec{T})\cap Range(\vec{T})=\cbrak{0}$.
    \item $Kernel(\vec{T}^2)\cap Range(\vec{T}^2)=\cbrak{0}$.
\end{enumerate}
\section{\textbf{Explanation}}
\renewcommand{\thetable}{1}
\begin{longtable}{|l|l|}
\hline
\endhead
$Range(\vec{T})$&It is column-space of linear operator $\vec{T}$.\\&\parbox{15cm}{\begin{align}
    \vec{T}(\vec{x})=\vec{v}
    \implies\vec{Ax}=\vec{v}
\end{align}}\\&where $\vec{x}$,$\vec{v}\in\vec{V}$ and columns of matrix $\vec{A}$ is the basis of column-space of linear\\&operator $\vec{T}$.\\
\hline$Kernel(\vec{T})$&It is null-space of linear operator $\vec{T}$.\\&\parbox{15cm}{\begin{align}
    \vec{T}(\vec{x})=0
    \implies\vec{Ax}=0
\end{align}}\\&where $\vec{x}\in\vec{V}$ and matrix $\vec{A}$ is same as before.\\
\hline$rank(\vec{T})$&\parbox{15cm}{\begin{align}
    rank(\vec{T})=rank(\vec{A})
\end{align}}\\
\hline$\vec{T}^2$&\parbox{15cm}{\begin{align}
    \vec{T}^2(\vec{x})&=\vec{A}^2\vec{x}\quad\quad\vec{x}\in\vec{V}\\
    rank(\vec{T}^2)&=rank(\vec{A}^2)
\end{align}}\\
\hline$\vec{A}$ and $\vec{A}^2$&The basis vectors of column-space of $\vec{A}$ and $\vec{A}^2$ are same.\\&The basis vectors of null-space of $\vec{A}$ and $\vec{A}^2$ are same.\\
\hline
\caption{Definitions and theorem used}
\label{deftab}
\end{longtable}
\newpage
\section{\textbf{Solution}}
\renewcommand{\thetable}{2}
\begin{longtable}{|l|l|}
\hline
\endhead
\textbf{Statement}&\textbf{Observations}\\
\hline
Given&$\vec{V}$ is a finite dimensional space over $\mathbb{R}$ and $T:\vec{V}\rightarrow\vec{V}$\\&\parbox{15cm}{\begin{align}
    rank(\vec{T})=rank(\vec{T}^2)\label{r}
\end{align}}\\&According to rank-nullity theorem.\\&\parbox{15cm}{\begin{align}
    dim(\vec{V})=rank(\vec{T})+nullity(\vec{T})\label{drn1}\\
    dim(\vec{V})=rank(\vec{T}^2)+nullity(\vec{T}^2)\label{drn2}
\end{align}}\\&from \eqref{drn1} and \eqref{drn2}. we get\\&\parbox{15cm}{\begin{align}
    \implies rank(\vec{T})+nullity(\vec{T})&=rank(\vec{T}^2)+nullity(\vec{T}^2)\\
    \implies nullity(\vec{T})&=nullity(\vec{T}^2)\label{n}
\end{align}}\\
\hline
\caption{Observations}
\label{obs}
\end{longtable}
\renewcommand{\thetable}{3}
\begin{longtable}{|l|l|l|}
\hline
\endhead
\textbf{Option}&\textbf{Solution}&\textbf{True/False}\\
\hline
1&From \eqref{n}, let&\\&\parbox{13cm}{\begin{align}
    nullity(\vec{T})&=nullity(\vec{T}^2)=n\label{p1}
\end{align}}&\\&Therefore, from table \ref{deftab} and \eqref{p1} we can say that both null space of&True\\&linear operator $\vec{T}$ and null space of linear operator $\vec{T}^2$ will have same n&\\& number of basis.&\\&\parbox{13cm}{\begin{align}
    \implies Kernel(\vec{T})=Kernel(\vec{T}^2)\label{res1}
\end{align}}&\\
\hline
2&From \eqref{r}, let&\\&\parbox{13cm}{\begin{align}
    rank(\vec{T})&=rank(\vec{T}^2)=r\label{p2}
\end{align}}&\\&Therefore, from table \ref{deftab} and \eqref{p2} we can say that both column space of&True\\&linear operator $\vec{T}$ and column space of linear operator $\vec{T}^2$ will have same r&\\& number of basis.&\\&\parbox{13cm}{\begin{align}
    \implies Range(\vec{T})=Range(\vec{T}^2)\label{res2}
\end{align}}&\\
\hline
3&From table \ref{deftab}, \eqref{p1} and \eqref{p2} we can that the r number of basis vectors&\\&of column space of linear operator $\vec{T}$ will form r-dimensional space which&\\&consists zero vector and n number of basis vectors of null space of linear&True\\&operator $\vec{T}$ will form n-dimensional space which consists zero vector.&\\&\parbox{13cm}{\begin{align}
    \implies Kernel(\vec{T})\cap Range(\vec{T})=\cbrak{0}\label{res3}
\end{align}}&\\
\hline
4&From table \eqref{res1}, \eqref{res2} and \eqref{res3}, we get&\\&\parbox{13cm}{\begin{align}
    \implies Kernel(\vec{T}^2)\cap Range(\vec{T}^2)=\cbrak{0}
\end{align}}&True\\
\hline
\caption{Solution}
\label{sol}
\end{longtable}
\end{document}
